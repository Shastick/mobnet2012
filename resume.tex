\documentclass[10pt, a4paper,twocolumn]{scrartcl}
\usepackage{amsmath}
\usepackage{amssymb}
\usepackage{multicol}

\topmargin=-0.9in      %
\evensidemargin=-0.8in     %
\oddsidemargin=-0.8in      %
\textwidth=8in        %
\textheight=10.7in       %
\headsep=0.1in         %

%Commands to format and shorten
\setlength{\columnseprule}{0.2pt}
\setlength{\parindent}{0cm}
\setlength{\parskip}{0ex plus 0.5ex minus 0.2ex}
\pagestyle{empty} 

\begin{document}

\section{Bianchi}

\subsection{probabilities}
$\pi$, probability of transmission, $p$, probability of collision,
$b_{i,k}$ stationary probability of state $i,k$:
\begin{align*}
p &= 1-(1-\pi)^{N-1} \\
\pi &= \frac{2}{ 1 + W_\textrm{min} + pW_\textrm{min}\sum^{m-1}_{k=0}(2p)^k}\\
	 &= \frac{2(1-p)}{(1-2p)(W_{min} + 1)+pW_{min}(1-(2p)^m)}\\
b_{i,k} &= \frac{CW_i - k}{CW_i} \cdot \left \{ \begin{array}{ll}
    (1-p) \sum_{j=0}^m b_{j,0} & i = 0\\
    p \cdot b_{i-1,0} & 0 < i < m\\
    p \cdot (b_{m-1,0} + b_{m,0}) & i = m
    \end{array} \right.
\end{align*}

\subsection{Saturation throughput}

\begin{align*}
 \tau &= \frac{E\lbrack\textrm{Payload Transmitted by user i in a slot time}\rbrack}{E[\textrm{Duration of slot time}]} \\ 
 &= \frac{P_\textrm{s}P_{\textrm{tr}}L}{P_\textrm{s}P_{\textrm{tr}}T_{\textrm{s}} + P_\textrm{tr}(1-P_\textrm{s})T_\textrm{c} + (1-P_\textrm{tr})T_\textrm{id}}, \\
 P_\textrm{s} &= \frac{N\pi (1-\pi)^{N-1}}{1-(1-\pi)^N}, \\
 P_\textrm{tr} &= 1-(1-\pi)^N, \\
 T_\textrm{s} &= t_\textrm{header} + t_\textrm{payload} + \textrm{SIFS} + t_\textrm{ACK} + \textrm{DIFS} + \sigma,\\
 T_\textrm{c} &= t_\textrm{header} + t_\textrm{payload} + \textrm{SIFS} + \sigma
\end{align*}

TODO : Times, and the Bianchi model's equation.

\section{Privacy Metrics}

\subsection{Entropy-Based Anonymity}
$A$ the anonymity set, $p_x$ the probability for an external observer that the action was performed by $x$:
\begin{equation*}
	\sum_{\forall x \in A} p_x \log(p_x)
\end{equation*}

\subsection{Entropy-Based Unlinkability}
$I_1,I_2$, sets of elements to be related, $p_r$, the probability two elements are related for an external observator:
\begin{equation*}
	\sum_{\forall R \subseteq I_1 \times I_2}p_r \log(p_r)
\end{equation*}

\section{TODO: RFID STUFF}
See exam 2010  ex 7 for an example.
\section{TODO: LOCATION PRIVACY STUFF}
\section{TODO: TCP setups in mobile networks}

\section{Trunk dimensioning}
For a trunk of $N$ channels, an offered load $A=\lambda E[X]$, $X$ the call duration, $Y$ the call arrival per sec $\sim$ Poisson($\lambda$) 
\begin{align*}
	P_\textrm{Blocking} &= P(\textrm{Drop a call because busy line})  \\
	&= \frac{A^N}{N!\sum^N_{i=0}(\frac{A^i}{i!})}
\end{align*}

Each channel carries the traffic 

\begin{equation}
	\rho = \frac{(1 - P_\textrm{blocking})A}{N}
\end{equation}

\paragraph{Cellular efficiency} $E = \frac{Conversations}{cells\times MHz}$

\section{Cellular Geometry: Hexagons}

\textbf{Area}: $A=1.5R^2\sqrt{3}$\\
\textbf{Distance btw. adjacent cells}: $d=\sqrt{3}R$

\subsection{Co-channel interference}
\begin{description}
\item[Co-channel reuse ratio]: $Q = \frac{D}{R} = \sqrt{3N}$ with $D$ the \textbf{distance} to the nearest co-channel cell, $R$ the \textbf{radius} of a cell and $N$ the \textbf{cluster size}.

\item[Signal to Interference ratio (SIR)]: $\textrm{SIR} = \frac{S}{I} = \frac{S}{\sum^{i_0}_{i=1}I_i}$. With $S$ the desired signal \textbf{power}, $I_i$ the \textbf{interference power} from the $i$th interfering co-channel base-station, $i_0$ the \textbf{number of co-channel} interfering cells.

\item[Signal to Interference plus Noise ratio (SINR)] : SINR $= \frac{S}{I + N_0}$

\item[Average received power $P_r$]: $P_r = P_0(\frac{d}{d_0})^{-\alpha}$ or \\ 
$P_r(\textrm{dBm}) = P_0(\textrm{dBm})-10\alpha\log(\frac{d}{d_0})$ with $P_0$ the power received from a small distance $d_0$ from the transmitter and $\alpha$ the path loss exponent.
	
\item[SIR in the corner of a cell]: $\frac{S}{I} = \frac{R^{-\alpha}}{\sum^{i_0}_{i=1}D_i^{-\alpha}}$

\item[First interfering layer approximation]: $\frac{S}{I} = \frac{(\frac{D}{R})^\alpha}{i_0} = \frac{(\sqrt{3N})^\alpha}{i_0}$ eg. $=(\frac{D}{R})^2\frac{1}{2}$ for two first layer interferers (cell divided into 3 sectors with directional antennas.)

\end{description}

\subsection{Capacity of a cellular network}
For $B_\textrm{t}$ the total allocated spectrum and $B_\textrm{c}$ the channel bandwidth: 
\begin{align*}
m = \frac{B_t}{B_c \frac{Q^2}{3}} = \frac{B_t}{B_c\left(\frac{6}{3^{\frac{\alpha}{2}}}\left(\frac{S}{I}\right)_\textrm{min}\right)^{\frac{2}{\alpha}}}
\end{align*}
For a cluster size $N$, $N = (i + j)^2 - ij$ for $i,j=0,1,2,\ldots$ and number of channels $C$ we have $m=\lfloor\frac CN\rfloor$

\subsection{CDMA Capacity: single cell case}
For the bitrate $R$, available bandwidth $W$, noise spectral density $N_0$, thermal noise $\eta$, received user signal (at base station) $S$, we have a possible number $N$ of users:
\begin{align*}
	N = 1 + \frac{W/R}{E_b/N_0} - (\frac{\eta}{S})
\end{align*}
With a duty cycle $\delta$ (Discontinuous transmission mode: takes advantage of
intermittent nature of speech):
\begin{align*}
	N = 1 + \frac1\delta\frac{W/R}{E_b/N_0} - (\frac{\eta}{S})
\end{align*}
And if we have $m$ sectors, the effective capacity becomes $mN$
\subsection{CDMA multiple cells}
\paragraph{Frequency reuse factor on the uplink} 
$f = \frac{N_0}{N_0 + \sum_iU_iN_{ai}}$ where $N_0$ = total interference power received from $N-1$ in-cell users, $U_i$ = number of users in the$i^\text{th}$ adjacent cell and $N_{ai}$ = average interference power from a user located in the $i^\text{th}$ adjacent cell

\paragraph{Average received power from users in adjacent cell}
$N_{ai} = \sum_j N_{ij}/U_i$ where $N_{ij}$ = power received at the base station of interest from the $j^\text{th}$ user in the $i^\text{th}$ cell

\subsection{Propagation modes}
\begin{multicols}{2}
	\begin{description}
		\item[Ground Wave] : $f \le 2$ Mhz
		\item[Sky Wave]
		\item[Line of Sight] : $f \ge 30$ Mhz
	\end{description}
\end{multicols}

\subsection{Line of sight equations}
Horizon distance $d[\textrm{km}]$ in \textbf{kilometers}, antenna height $h[\textrm{m}]$ and refraction adjustment factor $K = 4/3$:
\begin{description}
\item[Optical LOS]: $d = 3.57\sqrt{h}$
\item[Effective LOS]: $d = 3.57\sqrt{Kh}$
\item[Max LOS distance for two antennas :] $3.57(\sqrt{Kh_1}+ \sqrt{Kh_2})$
\end{description}

\section{Noise}
\paragraph{Thermal Noise}$N_0 = kT\quad(W/Hz)$

For signal power $S$, bitrate $R$, $k = 1.3806\cdot10^{-23} JK^{-1}$ the Boltzmann constant and $T$ the temperature: $\frac{E_b}{N_0} = \frac{S/R}{N_0} = \frac{S}{kTR}$

%Include slide 26 of A2 (throughput expressions)? 
\section{Wireless Misc Stuff}
\underline{Security Requirements}:
\textbf{Confidentiality, Authenticity, Replay Detection, Integrity, Access Control,Jamming Protection}

\subsection{Ad-hoc Netowrks} %D3 - slide 48
\paragraph{Upper Bound for the Throughput} 
If we have $n$ identical randomly located nodes each capable of transmitting $W$ bits/s. 
Then the throughput $\lambda(n)$ obtainable by each node for a randomly chosen destination is $\lambda(n) = \Theta\left(\frac W{\sqrt{n\log n}}\right)$

\subsection{Antennas \& Propagation}
Free space propagation, received power: $P_\textrm{R} = P_\textrm{T}\frac{A_\textrm{R}}{4\pi d^2}\eta_\textrm{R}$ with $\eta_\textrm{R}$ an efficiency parameter, $A_\textrm{R}$ the receiving antenna area.
\\
Focusing capability, depends on size in wavelength $\lambda$:  
\\$G_\textrm{T} = 4\pi\eta_\textrm{T}A_\textrm{T}/\lambda^2$ \\
Directional emitter, received power: $P_\textrm{R} = P_\textrm{T}G_\textrm{T}\frac{A_\textrm{R}}{4\pi d^2}\eta_\textrm{R}$

Free space received power: $P_\textrm{R} =  P_\textrm{T}G_\textrm{T}G_\textrm{R}(\frac{\lambda}{4\pi d})^2$

Loss: $L = \frac{P_T}{P_R} = \frac{(4\pi d)^2}{G_RG_T\lambda^2} $

$ c = 3 \cdot 10^8 $

Parabola: $G = \frac{7A}{\lambda^2}$

\underline{\textbf{Mobnet Decibels}}:
$B = 10\log(\frac{P}{P_0})$
\subsection{Free Space Loss}
%TODO PICKUP HERE
Free space loss, ideal isotropic antenna:
$$ \frac{P_t}{P_r} = \frac{(4\pi d)^2}{\lambda^2} = \frac{(4\pi fd)^2}{c^2} $$
Free space loss equation can be recast:
$$L_{DB} = 10\log \frac{P_t}{P_r} = 20 \log(f) +20\log(d) - 147.56 dB$$
Free space loss accounting for gain of other antennas: 
$$\frac{P_t}{P_r} = \frac{(4\pi d)^2}{G_rG_t\lambda^2} = \frac{(cd)^2}{f^2A_rA_t}$$
$G_t$ = gain of transmitting antenna\\
$A_r$ = effective area of receiving antenna

\section{Ad-hoc Netowrks} %D3 - slide 48
\paragraph{Upper Bound for the Throughput} 
f we have $n$ identical randomly located nodes each capable of transmitting $W$ bits/s. 
Then the throughput $\lambda(n)$ obtainable by each node for a randomly chosen destination is $\lambda(n) = \Theta\left(\frac W{\sqrt{n\log n}}\right)$


\section{Wireless Misc Stuff}
\underline{Security Requirements}:
\textbf{Confidentiality, Authenticity, Replay Detection, Integrity, Access Control,Jamming Protection}


\subsubsection{ALOHA}
\begin{equation}
P(k \text{ transm. in } 2X\text s) = \frac{(2G)^k}{k!} e^{-2G}
\end{equation}

\begin{equation}
S = G \cdot P(0) = Ge^{-2G}
\end{equation}

\subsubsection{Slotted ALOHA} 
%C'est pas slotted alpha ça? 
Probability of $k$ packets generated during a slot:

\begin{equation}
P(k) = \frac{G^ke^{-G}}{k!}
\end{equation}

Throughput:

\begin{equation}
P(1) = Ge^{-G}
\end{equation}

\subsubsection{CSMA}
\paragraph{Non-persistent} If channel is busy, directly run back off algorithm
\paragraph{p-persistent} If it is busy, they persist with sensing until the channel becomes idle. If it is idle:\\
- With probability $p$, the station transmits its packet\\
- With probability $1-p$, the station waits for a random time and senses again

\subsection{DOMINO Cheating detection}
\begin{tabular}{|p{4cm}|p{5.2cm}|}
  \hline
  Cheating Method & Detection Test \\\hline
  Frame scrambling & Number of retransmissions \\
  Oversized NAV1 & Comparison of the declared and actual NAV values\\
  Transmission before DIFS & Comparison of the idle time after the last ACK with DIFS \\
  Backoff manipulation & Actual Backoff/ Consecutive Backoff \\
  Frame scrambling with MAC forging & Periodic dummy frame injection\\
   \hline
\end{tabular}

\subsection{Forward Error Correction (FEC)}
Redundancy in packets to allow limited error correction at the receiver: used in 802.11a (Convolutional), HSDPA (Turbo Codes) and 802.11n (LDPC).

\section{TCP}
\subsection{Standard}
\paragraph{Tahoe}
\paragraph{Reno}
\paragraph{New Reno}
\subsection{Mobile}
\paragraph{Indirect TCP (I-TCP)} Connection split at FA. Standard TCP on the wire line, wireless optimized TCP on the wifi side: shorter timeout, faster retransmission. Loss of end-to-end semantics, security issues.
\paragraph{Mobile TCP (M-TCP)} Split connection at FA. Monitor packets, if a disconnect is detected, report receiver window $= 0$: sender will go into persist mode and doesn't timeout or modify his congestion window. Preserves end-to-end semantics. Disadv.: wifi losses propagate to the wire network, link-errors pkt loss must be resent by sender, security issues. \underline{Summary}: only handles mobility errors, no transmission errors.

\paragraph{Snooping-TCP} TCP-aware link layer: Split connections, FA buffers and retransmits segments, does not ACK buffered packets (preserves end-to-end semantics).

\paragraph{Transaction oriented TCP (T-TCP)} TCP phases: connection setup, data transmission, connection release. T-TCP combines does steps and only 2-3 packets are needed for short messages. Efficient for single packet transactions, but requires TCP modifications on all hosts.
\end{document}



